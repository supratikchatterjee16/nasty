\documentclass[16pt,a4paper,twocolumn,conference]{IEEEtran/IEEEtran}
\usepackage{cite}
\title{News Analysis and Summarization from RSS feeds}
\author{
\IEEEauthorblockN{Supratik Chatterjee\\}
\IEEEauthorblockA{
B.Tech, CSE, Third Year,\\
SRM IST,\\
Vadapalani, Chennai,\\
Tamil Nadu, India.\\
RA1511003040196
}\\
\and
\IEEEauthorblockN{Shubham Bagadiya\\}
\IEEEauthorblockA{
B.Tech, CSE, Third Year,\\
SRM IST,\\
Vadapalani, Chennai,\\
Tamil Nadu, India.\\
RA1511003040258
}\\
\and
\IEEEauthorblockN{Sanjeev Siva\\}
\IEEEauthorblockA{
B.Tech, CSE, Third Year,\\
SRM IST,\\
Vadapalani, Chennai,\\
Tamil Nadu, India.\\
RA1511003040193
}\\
}
\setlength\parskip{0.5cm}
\begin{document}
\maketitle
\begin{abstract}
\large\noindent In the modern day field of media, news is majorly affected by the social platforms such as Facebook, Twitter, etc. It becomes necessary to understand the facts of an event, and to quantify or relate them to understand it. Most humans do not have the dedication or time to study all the events at all the time, as time is always limited. However, an aggregation and analysis system on a computer could do it for them.\\
\end{abstract}

\begin{IEEEkeywords}
\centering
Natural Language Processing, Web Scraping, Text Mining, news, Co-occurence Analysis, Information Centralization, Machine Learning, Clustering, News Aggregation
\end{IEEEkeywords}

\section{Introduction}
The aim of the project is to centralize information collected from news websites for connected and summarized reading. This aims to remove co-occurance of facts, and find discrepancies of figures, so that one might deal with them as an outlier. The rest of the data must be catalouged and indexed such that they may be referenced easily. When this is hooked up with a Machine Learning tool, should be able to predict outcomes, provided the dataset given to it has been collected for a duration long enought for it to work on developing itself.\\
\par Collection of information has to be from the several RSS feeds found online. Using those as the data point one can fetch the information and centrlize the resources for further analysis and prediction. This project will be an exercise in centrality, distributed computing, natural language processing, data mining, clustering and machine learning.\\
\section{Literature Review}
There exists no such tool that analyzes and aggregates the feeds for others. However, there are applications made to organize the feeds for easy viewing. There are several problems for such an application. These problems areas follows :
\begin{enumerate}
\item Articles are repeated
\item No summarization or focus on facts
\item No check on the feeds
\item No interrelation between points of interest
\item Hard to look up
\item Impossible to keep track
\end{enumerate}
 
\end{document}
